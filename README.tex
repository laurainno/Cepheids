******************************************************************************************************************
Mock catalogue of Classical Cepheids in PS1
******************************************************************************************************************

- Step 1: Producing normalized light-curves templates in PS1 grizy bands

        1. Calibrating sample
            I adopted the sample of 131 Cepheids from Monson & Pierce 2011 for which I already have photometric data
            in the UBVI bands from literature.
            I performed a binning in period: I determine different templates light curves
            for Cepheids with period
            bin2:  3 days < P $\leqslant$5 days 
            bin3:  5 days < P $\leqslant$7 days 
            bin4:  7 days < P $\leqslant$9.5 days 
            
            In fact, the shape of the light curves changes with the mass (i.e. the period) 
            and the bump is on the decreasing branch in bin3, but already
            closer to the maximum light in bin4.
            
            Note that if we want to compute templates for P<3 days,
            we need to adopt SMC light curves, since there are no Galactic
            calibrating Cepheids in this period range.
            
            The templates are computed by obtained normalized merged light curves
            of the calibrating Cepheids in the specific bin.
            The typical rms of the merged light curves is 0.03--0.05 mag.
            This rms comes from photometric errors on the 
            calibrating Cepheids individual light curves and by the finite binning in period
            (intrinsic effect) of the merging.
            
        2. Photometric system transformation
        
            I adopted the color transformation in Tonry et al. 2012 to convert the 
            BVIJH calibrating light curves in the PS1 photometric system (their Table 6).
            I used the equation 6 in the linear form, i.e y= B0+B1*x
        
        
        3. Amplitudes
        
            I obtained amplitudes in the different PS1 bands for the adopted calibrating Cepheids.
            Amplitudes as a function of Log P are plotted in Fig. PS1_ampl_logp.jpg 
            To predict the expected amplitude for a given period 
            we can adopt the median value in the period range 3 days <P<9.5 days.
            The values are given in Table ampl_logp_table.pdf
            
            Amplitudes in individual bands are necessary to rescale the normalized light-curve-templates.


        4. Deriving absolute luminosities 
            I adopted theoretical models to determine PL relations in the PS1 bands.
            The relations are given in Table PS1_pl_rel.pdf in the form 
            $$Mag_{band}= a_{band} + b_{band} * Log P $$.
            I also derive a theoretical PL relation to predict mean magnitude in the 2MASS $K_{\rm{S}}$ band.
            I used instead the relation from Marengo et al. 2009 to predict the W1 mean magnitude.
     
            Note that this relations is derived for LMC Cepheids in the Spitzer 3.4 um band. ***


- Step 3: Choosing selected line-of-sights in PS1 real data
        
        1. LOS selection
                I selected the following line-of sights:
                 1) l=22.0     b=3.0
		2) l=27.0     b=3.3
		3) l=36.6     b=0.007
		4) l=50.8     b=0.7
		5) l=61.23   b=0.1
		6) l=65.5     b=-1.9
		7) l=50.08   b=0.7
		8) l=76.021 b=-0.677
		9) l=91.96   b=-0.56
		10) l=122      b=0.00
		11) l=142.9   b=1.60
		12) l=156.6   b=0.047
		13) l=202      b=1.6                   
        2. Epochs sampling
                To obtain the epochs sampling for the selected LOS, I choose an object which is not saturated and is not too faint.
                This choice is performed by selecting a random objet with psf_inst_sigma ~0.01
                and getting its obs_MJD.
            
        3. Reddening from Green et al. 2015
                I use the first raw on the 3D reddening map downloaded from the website http://argonaut.skymaps.info/ 
                for the given galactic coordinates.
    
        
- Step 4: Producing the final mock catalog
        
        1. Period sampling
            I have chosen a period sampling of 0.5 days
        
        2. Distance/ reddening sampling
                I  repeat the same simulated light curve for each value of  dm and ebv  from the reddeing table
                which will not produce saturated observations in the final magnitude.
        
        *** Note that at the moment if the brightest observation of the light curve saturates, the light curve is entirely skipped. 
        In the same way, if the faintest observation is fainter than the detection limit, than the entire light curve is skipped.
        This can be improved, i.e. we can include all the not-saturated and detected observations.***
        
        3. Error modeling
        In order to determine the photometric error on the observed magnitude I did as follows:
        - I calibrated the observed magnitudes in the real PS1 data by apllying Nina/Eddie python procedure:
        ubercal_flat_bigflat_chmask
        - I assumed a photometric uncertainty given by mag_err= sqrt((1.3*psf_inst_sigma)^2+ 0.015^2)
        - I fitted the log(mag) vs log(err) with a third-order polynomial 
        - I used the fitted relations to predict the error at a given magnitude in the simulated data. 
        - For each simulated observation, I perform a random extraction from a gaussian distribution 
        with a sigma given by the predicted photometric error and the uncertainty due to the empirical template (~0.05 mag). 
        - I recompute the expected error by adopting the polynomial relations to the new value obtained by including random error effects
        
        
        






